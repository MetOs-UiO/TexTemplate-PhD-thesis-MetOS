\color{red}
\chapter[Summary and concluding remarks]{{\color{red}\fontsize 90 :}Summary and concluding remarks}
\label{ch:conclusion}

This chapter provides a short summary with specific answers to the sub objectives, in addition to a discussion on how the main objective were achieved and which relevance the findings from thesis has to the larger scientific community. 

\section{Summary of answers to the sub objectives}
\begin{enumerate}
        \item \textbf{To investigate, apply and improve the nudging facilities in NorESM.}\\
	      Results carried out with the old nudging setup in NorESM were found not to be able to represent the ones carried out with a version of the model running with free meteorlogy. This was improved by reducing the number of constrained variables, in addition to reducing the nudging intensity (increasing the relaxation time scale). The method of nudging was applied in all papers of the thesis. 
        \item \textbf{To develop and apply a reanalysis dataset for nudging in NorESM, enabling comparisons of modelled aerosol-cloud interactions to observations.} \\
	      A nudging dataset suited for NorESM, based on 19 years of reanalysis data from ERA-Interim, was generated. This was applied in Paper III and IV to be able to compare modelled cloud responses to the Holuhraun eruption to those of satellite retrievals. It was also used in Paper I to show that various modelled aerosol, cloud and radiation properties carried out by a nudged version of the model were mostly similar to the same properties carried out by a version of the model running with free meteorlogy. 
	\item \textbf{To investigate the impact of historical oxidant changes on ERFaci, and propose improved treatment of oxidants when modelling ERFaci.}\\
	      Paper II found that including oxidant changes between PI and PD is very important for the magnitude of the modelled ERFaci (+0.25 Wm$^{-2}$, 19 \% change), mainly because of its impact on the lifetime of the precursor gases, affecting both where, when and how aerosol formation, aerosol growth and cloud droplet activation occur. When modelling changes in aerosol-cloud interactions between different eras, using prescribed oxidant fields, the findings of this thesis suggest that not only aerosols and aerosol precursor gases should be switched between the two simulations, but also the oxidant fields. Aerosol precursor gases should be exposed to oxidants of its era. 
	\item \textbf{To identify problems linked to global modelling of rapid cloud adjustments by taking part in an intercomparison study that uses a recent volcanic eruption as a testbed to evaluate model performances of simulated aerosol-cloud interactions.}\\
	    While satellite retrievals from the time og the eruption indicate that LWP are well buffered against aerosol changes, global models in Paper III did not find the same, indicating their inadequacy of simulating rapid cloud adjustments properly. Regarding the first aerosol indirect effect, global models were able to capture its sign and magnitude, as the satellite retrievals, enhancing our confidence in it beeing both negative and of importance. 
	\item \textbf{To explore impacts on rapid cloud adjustments by implementing size-dependency on two processes that can reduce cloudiness when aerosol concentration increase (evaporation and entrainment).}\\
	    Paper IV found that implementations of size-dependency on entrainment and evaporation processes in global models may not result in strong suppressions of initial increases in LWP when aerosol concentrations increase because changes in droplet sizes are too small, or because feedback processes linked to stability changes can be counteracting. This highlighs more complex aspects of the previously proposed buffering mechanisms of rapid cloud adjustments that has been in in focus before.   
\end{enumerate}

\section{Overskrift her}
The overall objective of this thesis was to contribute to the work towards enhancing confidence in aerosol effects on climate through improved global modelling of aerosol-cloud interactions by the using NorESM. The answers to the sub objectives in the previous sections shows that global modelling of aerosol-cloud interactions is improved by the suggestion of new guidelines for the treatment of the oxidants and improved nudging facilities. The implementations of buffering mechanisms explored in Paper IV did not manage to improve the model treatment of rapid cloud adjustments, to better fit with the constraints revealed in Paper III. Nonetheless, it provided information about the model respons to these implementations, which could be useful for later studies further exploring these mechanisms and possibly come up with model improvements. 





\color{black}